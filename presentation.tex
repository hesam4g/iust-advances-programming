\documentclass{bidipresentation}
\pagestyle{pres}
\usepackage{color}
\usepackage{graphicx}

\definecolor{backgroundcolor}{rgb}{.5,0,0}
\definecolor{backgroundcolor1}{rgb}{0,0.5,0}
\definecolor{backgroundcolor2}{rgb}{0,0,0.5}
\definecolor{backgroundcolor3}{rgb}{.5,.5,0}
\definecolor{backgroundcolor4}{rgb}{.5,0,0.5}
\definecolor{backgroundcolor5}{rgb}{0,0.5,0.5}
\definecolor{textcolor}{rgb}{1,1,1}
\definecolor{headcolor}{rgb}{1,1,0}
\pagecolor{backgroundcolor}
\color{textcolor}
\usepackage{xepersian}
\settextfont{Neirizi}
\makeatletter
\def\item{%
  \@inmatherr\item
  \@ifnextchar [\@item{\@noitemargtrue \@item[\textcolor{headcolor}{\@itemlabel}]}}
\makeatother
\begin{document}

%\begin{figure}[h]
%\centering 
%\includegraphics[width=10mm]{procedural1.png}
%\caption{چارچوب کلی شیوه‌ی پیشنهادی.}\label{fig:OneLR_oneHR}
%\end{figure}

%\begin{equation}
%f(x)=\frac{a_0}{2}+\sum_{n=1}^\infty\left(a_n\cos nx+b_n\sin nx\right)
%\end{equation}


\color{textcolor}
\begin{titlepage}
\centering
\distance{1}
{\color{headcolor}
\Huge \bfseries طراحی شی گرا با استفاده از \\ C++ \par
}
\vspace{1.3ex} \large
برنامه نویسی پیشرفته\\[2ex]دانشگاه علم و صنعت ایران
\distance{2}
\end{titlepage}

\begin{plainslide}[مروری بر شگردهای برنامه نویسی\LTRfootnote{ Programming Tecnhiques}]

\begin{itemize}
\item برنامه سازی بی ساختار\LTRfootnote{ Unstructured programming}
\item برنامه سازی روندگرا\LTRfootnote{ procedural programming}
\item برنامه سازی بسته ای\LTRfootnote{ modular programming}
\item برنامه سازی شی گرا\LTRfootnote{ object-oriented programming}
\end{itemize}

\end{plainslide}

\begin{plainslide}[ مسیر چینی ساختار تهی]
لاک پشت عزیز، 3 قدم به جلو برو، بعد یک قدم به سمت را  بپیچ، بعد ببین آیا  پایت به چیزی میخورد؟ اگر خورد ببین آیا آن چیز پله است در غیر اینصورت 5 قدم به سمت جلو برو و بعد ببین آیا پایت به چیزی میخورد؟ اگر خورد ببین آیا آن چیز پله است. ......
\end{plainslide}

\begin{plainslide}[ مسیر چینی ساختار تهی]
\underline{لاک پشت عزیز}، 3 قدم به جلو برو، بعد یک قدم \underline{به سمت را  بپیچ}، بعد ببین 

\underline{آیا  پایت به چیزی میخورد}؟ اگر خورد ببین \underline{آیا آن چیز پله است} در غیر اینصورت 5 قدم به سمت جلو برو و بعد ببین \underline{آیا پایت به چیزی میخورد}؟ اگر خورد ببین \underline{آیا آن چیز پله است.} ......
\end{plainslide}

\begin{plainslide}[ مسیر چینی ساختار تهی]
\begin{itemize}
\item برنامه اصلی به صورت مستقیم بر داده های عمومی اثرگذاری میکردند.
\item برنامه ها بسیار بلند میشدند.
\end{itemize}
\end{plainslide}

\pagecolor{backgroundcolor1}

\begin{plainslide}[فرآیندگرایان ...]
\begin{itemize}
\item پایت به چیزی خورده است: دیگر نمیتوانی قدم برداری
\item آیا پله است: اگر سنگی بود و ارتفاع آن حدودا 30 سانت بود و ...
\item بررسی کردن رسیدن به پله: آن قدر به جلو برو تا پایت به چیزی بخورد، ببین آیا آن چیز پله است؟
\item 5 قدم به جلو برو، بررسی کن آیا به پله رسیدی؟ به سمت راست بپیچ، بررسی کن آیا به پله رسیدی ....
\end{itemize}
\end{plainslide}

\begin{plainslide}[فرآیندگرایان ...]
\begin{figure}[h]
\centering 
\includegraphics[width=20mm]{Images/procedural.PNG}
\caption{فرآیند اجرای برنامه}\label{fig:1}
\end{figure}
\begin{itemize}
\item واحدهای معنادار فرآیندی
\item خوانایی
\item فراخوانی سلسله مراتبی
\end{itemize}
\end{plainslide}

\begin{plainslide}[فرآیندگرایان ...]
\begin{figure}[h]
\centering 
\includegraphics[width=45mm]{Images/procedural1.png}
\caption{ساختار سلسله مراتبی}\label{fig:2}
\end{figure}
\end{plainslide}

\pagecolor{backgroundcolor2}

\begin{plainslide}[بسته های رفتاری ...]
\begin{itemize}
\item بسته رفتارهای ورزشی لاکپشت
\begin{itemize}
\item بدو
\item پروانه بزن
\item  پشتک وارون بزن
\end{itemize}
\item  بسته رفتارهای دانشجویی
\begin{itemize}
\item باز هم بدو
\item خر بزن
\item  درس پاس کن
\end{itemize}
\end{itemize}
\end{plainslide}

\begin{plainslide}[بسته های رفتاری ...]
\begin{figure}[h]
\centering 
\includegraphics[width=60mm]{Images/modular.png}
\caption{ساختار بسته ها و داده هایشان}\label{fig:2}
\end{figure}
\end{plainslide}


\begin{plainslide}[استفاده از ساختمان داده ها]
\begin{itemize}
\item لیست ها
\item دیکشنری ها و ...
\end{itemize}
مجموع نمرات یک دانش آموز، تمامی مشخصات یک آدم در یک دیکشنری و ....
\end{plainslide}

\pagecolor{backgroundcolor3}

\begin{plainslide}[تا اینجا چه داریم؟]
\begin{itemize}
\item بسته بندی داده ها در ساختمان داده ها
\item تعریف رفتارها در بسته های جدا
\end{itemize}
\end{plainslide}

\begin{plainslide}[دنبال چه چیزی بودیم؟]
\begin{itemize}
\item طراحی یک مفهوم خارجی
\item شبیه سازی یک اتفاق در دنیای خارج
\item انجام مجموعه ای از اعمال در دنیای مجازی دیجیتال متناسب با نیاز تعریف شده
\end{itemize}

بهترین روشی که میتوانیم یک مساله بیرونی را بفهمیم و برای پاسخ آن الگوریتمی پیشنهاد دهیم چیست؟
\end{plainslide}

\pagecolor{backgroundcolor4}
\begin{plainslide}[توصیف شی گرا]
\begin{enumerate}
\item هر چیزی شی است
\item یک برنامه عبارتست از مجموعه ای از اشیاء که با پیغام با همدیگر صحبت میکنند
\item حافظه هر شی تشکیل شده است از اشیاء دیگری که در آن هستند.
\item هر شی از یک نوع است
\item همه اشیاء از یک نوع مشترک پیغام های مشترک دریافت میکنند
\end{enumerate}
\end{plainslide}

\begin{plainslide}[عالم مُثُل!!!!]
\begin{itemize}
\item هر شی یک وجه مُثُلی دارد
\item نوع: مجموعه ای از مشخصات و رفتارهای مشترک بین اشیاء
\end{itemize}
\begin{figure}[h]
\centering 
\includegraphics[width=60mm]{Images/light.png}
\caption{لامپ}\label{fig:2}
\end{figure}
\end{plainslide}

\pagecolor{backgroundcolor3}
\begin{plainslide}[واما لامپ]
\lr{
\\
Light myRoom1Light;\\
myRoom1Light.on();\\
\\
Light myKitchenLight;\\
myKitchenLight.off();\\
}
\end{plainslide}

\begin{plainslide}[دانشجو، استاد در سمینار]
\begin{figure}[h]
\centering 
\includegraphics[width=90mm]{Images/classDiagramInheritance.jpg}
\end{figure}
\end{plainslide}

\pagecolor{backgroundcolor}

\begin{plainslide}[پیاده سازی مخفی]
\begin{itemize}
\item افراد در قبال یک کلاس: سازنده، استفاده کننده
\item \lr{public,private,protected}
\end{itemize}
\end{plainslide}

\begin{plainslide}[قابلیت استفاده مجدد]
\begin{figure}[h]
\centering 
\includegraphics[width=90mm]{Images/generic_car_cutaway_drivetrain.jpeg}
\end{figure}
\end{plainslide}

\begin{plainslide}[قابلیت استفاده مجدد]
\begin{figure}[h]
\centering 
\includegraphics[width=90mm]{Images/has.png}
\caption{این از آن دارد}
\end{figure}
\end{plainslide}

\pagecolor{backgroundcolor1}
\begin{plainslide}[قابلیت استفاده مجدد]
\begin{figure}[h]
\centering 
\includegraphics[width=30mm]{Images/car-big.jpg} \\
\includegraphics[width=50mm]{Images/Cars_full.JPG}
\end{figure}
\end{plainslide}

\begin{plainslide}[قابلیت استفاده مجدد]
\begin{figure}[h]
\centering 
\includegraphics[width=60mm]{Images/derived.png} 
\caption{این از نوع آن است}
\end{figure}
\end{plainslide}

\pagecolor{backgroundcolor2}
\begin{plainslide}[قابلیت استفاده مجدد]
\begin{figure}[h]
\centering 
\includegraphics[width=70mm]{Images/interface.png} 
\caption{این مانند آن است}
\end{figure}
\end{plainslide}

\pagecolor{backgroundcolor3}
\begin{plainslide}[تکلیف]
\begin{enumerate}
\item \ltr{UML Class Diagrams}
\item \rtl{طراحی اشیاء سیستم آموزشی یک مدرسه راهنمایی و آپلود عکس آن در سایت}
\end{enumerate}
\end{plainslide}

\pagecolor{backgroundcolor4}
\begin{plainslide}[\LaTeX]
این ارائه با استفاده از زی پرشین که یکی از افراد خانواده \TeX میباشد آماده شده است.
\end{plainslide}

\end{document}
